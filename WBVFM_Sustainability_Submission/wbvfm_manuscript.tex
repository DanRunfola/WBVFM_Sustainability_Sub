%  LaTeX support: latex@mdpi.com 
%  In case you need support, please attach all files that are necessary for compiling as well as the log file, and specify the details of your LaTeX setup (which operating system and LaTeX version / tools you are using).

% You need to save the "mdpi.cls" and "mdpi.bst" files into the same folder as this template file.

%=================================================================
\documentclass[sustainability,article,submit,moreauthors,pdftex,10pt,a4paper]{mdpi} 

%=================================================================
\firstpage{1} 
\makeatletter 
\setcounter{page}{\@firstpage} 
\makeatother 
\articlenumber{x}
\doinum{10.3390/------}
\pubvolume{xx}
\pubyear{2016}
\copyrightyear{2016}
%\externaleditor{Academic Editor: name}
\history{Received: date; Accepted: date; Published: date}

%------------------------------------------------------------------
% The following line should be uncommented if the LaTeX file is uploaded to arXiv.org
%\pdfoutput=1

%=================================================================
% Add packages and commands here. The following packages are loaded in our class file: fontenc, calc, indentfirst, fancyhdr, graphicx, lastpage, ifthen, lineno, float, amsmath, setspace, enumitem, mathpazo, booktabs, titlesec, etoolbox, amsthm, hyphenat, natbib, hyperref, footmisc, geometry, caption, url, mdframed

\usepackage{tablefootnote}

%=================================================================
%% Please use the following mathematics environments: Theorem, Lemma, Corollary, Proposition, Characterization, Property, Problem, Example, ExamplesandDefinitions, Remark, Definition
%% For proofs, please use the proof environment (the amsthm package is loaded by the MDPI class).

%=================================================================
% Full title of the paper (Capitalized)
\Title{A Top-Down Approach to Estimating Spatially Heterogeneous Impacts of Development Aid on Vegetative Carbon Sequestration}

% If this is an expanded version of a conference paper, please cite it here: enter the full citation of your conference paper, and add $\dagger$ in the end of the title of this article.
%\conference{}

% Authors, for the paper (add full first names)
\Author{Daniel Runfola $^{1}$\thanks{dsmillerrunfol@wm.edu}~, 
Ariel BenYishay $^{1}$\thanks{abenyishay@wm.edu}~, 
Jeff Tanner $^{2}$\thanks{jtanner@worldbank.org}~~, 
Graeme Buchanan $^{3}$\thanks{graeme.buchanan@rspb.org.uk}~~, 
Jyoteshwar Nagol$^{4}$\thanks{jnagol@umd.edu}~~, 
Matthias Leu $^{5}$\thanks{mleu@wm.edu}~~, 
Seth Goodman $^{1}$\thanks{smgoodman@wm.edu}~~~, 
Rachel Trichler $^{1}$\thanks{rbtrichler@wm.edu} ~~ and 
Robert Marty $^{1}$\thanks{ramarty@email.wm.edu}}
% Authors, for metadata in PDF
\AuthorNames{Daniel Runfola, Ariel BenYishay, Jeff Tanner, Graeme Buchanan, Jyoteshwar Nagol, Matthias Leu, Seth Goodman, Rachel Trichler and Robert Marty}

% Affiliations / Addresses (Add [1] after \address if there is only one affiliation.)
\address{%
$^{1}$ \quad Institute for the Theory and Practice of International Relations, The College of William and Mary \\
$^{2}$ \quad Independent Evaluation Group, World Bank \\
$^{3}$ \quad Center for Conservation Science, Royal Society of Birds \\
$^{4}$ \quad Global Land Cover Facility, University of Maryland \\
$^{5}$ \quad Department of Biology, The College of William and Mary \\}

% Contact information of the corresponding author
\corres{Correspondence: dsmillerrunfol@wm.edu; Tel.: +1-508-316-9109}

% Current address and/or shared authorship


% Simple summary
%\simplesumm{}

% Abstract (Do not use inserted blank lines, i.e. \\) 
\abstract{Since 1945, over \$4.9 trillion dollars of international aid has been allocated.
To date there have been no estimates of the regional impact of this aid on the carbon cycle. We apply a geographically explicit matching method to estimate the relative impact of large scale World Bank projects implemented between 2000 and 2010 on sequestered carbon, using a novel and publicly available data set of 61243 World Bank project locations. Considering only carbon sequestered due to fluctuations in vegetative biomass caused by World Bank projects, we illustrate the relative impact of World Bank projects on carbon sequestration across  development sectors. We use this information to illustrate the geographic variation in apparent effectiveness of environmental safeguards implemented by the World Bank. We argue that sub-national data can help identify geographically heterogeneous impact effects, and highlight the methodologic barriers which still exist.}

% Keywords
\keyword{Carbon Sequestration, Causal Identification, Heterogeneous Effects, Human Environment Interactions, International Aid}

% The fields PACS, MSC, and JEL may be left empty or commented out if not applicable
%\PACS{J0101}
%\MSC{}
%\JEL{}
%\AMS{}

%%%%%%%%%%%%%%%%%%%%%%%%%%%%%%%%%%%%%%%%%%
% Only for the journal Data:

%\dataset{DOI number or link to the deposited data set in cases where the data set is published or set to be published separately. If the data set is submitted and will be published as a supplement to this paper in the journal Data, this field will be filled by the editors of the journal. In this case, please make sure to submit the data set as a supplement when entering your manuscript into our manuscript editorial system.}

%\datasetlicense{license under which the data set is made available (CC0, CC-BY, CC-BY-SA, CC-BY-NC, etc.)}

%%%%%%%%%%%%%%%%%%%%%%%%%%%%%%%%%%%%%%%%%%
% For Conference Proceedings Papers: add the conference title here
%\conferencetitle{}

%\setcounter{secnumdepth}{4}
%%%%%%%%%%%%%%%%%%%%%%%%%%%%%%%%%%%%%%%%%%
\begin{document}

%%%%%%%%%%%%%%%%%%%%%%%%%%%%%%%%%%%%%%%%%%
\section{Introduction}

Global recognition of the dual challenges of international development and the mitigation of environmental change is resulting in a large redirection of resources towards developing countries. The United States alone has pledged nearly \$4 billion dollars of aid to mitigate climate vulnerability over the next 4 years, and the Paris convention urged donors to target \$100 billion annually by the year 2020 (\cite{royal_united_2015}). 
Coupled with increasing pressure from recipient nations, donors have and continue to introduce more stringent environmental safeguards on development projects of all types, including requirements for compliance with national and international regulations, environmental management plans, reforestation goals, and others (\cite{nielson_delegation_2003}, \cite{gutner_explaining_2005}).
\par
Despite these shifts in the policies of donor agencies, there is a gap in empirical studies examining the impacts of these policies on environmental outcomes. This letter aims to illustrate an approach to overcoming this gap, as well as highlight many of the remaining methodological challenges. We specifically examine the impact of large scale World Bank projects on vegetation and subsequent changes in carbon sequestration, leveraging a novel and publicly available data set of 61243 World Bank project locations (see Figure \ref{WBLocs})\footnote{http://aiddata.org/level1/geocoded/worldbank} in conjunction with long-term satellite data quantifying vegetative biomass and a number of spatially-referenced control variables (see Table \ref{data_source_table}).
\par
We find that while the overall impact of large scale World Bank projects on carbon sequestration appears to be positive, considerable temporal and spatial variation exists in these impacts. We illustrate the advantages and limitations of a geographically explicit approach to estimating the causal effects of development aid projects, and outline a number of topics for further research. Specifically, we discuss the need for hard assumptions of model independence in geographically explicit models to enable causal estimates, the concomitant limitations in interpretation this necessitates, and possible pathways forward to overcome this critical limitation. Finally, we introduce an enhanced, publicly available data set of the global, spatially-explicit distribution of World Bank activities encompassing projects initiated between 1995 and 2014.

\begin{figure}[H]
\centering
\includegraphics[width=1\textwidth, trim=0cm 6cm 0cm 2cm]{WLocs.png}
\caption{World Bank IDA and IBRD Project Locations.}
\label{WBLocs}
%\vspace{25pt}
\end{figure}  

\begin{table}[H]
\caption{Data sources used in this analysis.}\label{data_source_table}
\centering
\begin{tabular}[h]{|p{4.5cm}||p{7cm}|}
\hline
\multicolumn{2}{|c|}{\texttt{Data Sources}} \\
\hline
\textbf{Data Name} & \textbf{Source} \\
\hline
World Bank Geolocations & AidData\begin{math}^{1}\end{math} \\
\hline
Gridded Population of the World & Center for International Earth Science Information Network\tablefootnote{http://sedac.ciesin.columbia.edu/data/collection/gpw-v3/sets/browse} \\
\hline
Nighttime Lights & Defense Meteorological Satellite Program\tablefootnote{Stable Lights retrieved from http://ngdc.noaa.gov/eog/dmsp.html}\\
\hline
Precipitation and Temperature & University of Delaware (\cite{willmott_terrestrial_2001})\tablefootnote{Variables derived from these product included the average precipitation (P) and temperature (T) before a project was implemented (from 1992), the linear trend in P and T from 1992 to the project implementation, the average temperature from the date the project was implemented until the end of the temporal record(2012), and the post-project trend through 2012. Absolute measurements of each variable were also retained. } \\
\hline
Urban Travel Time & European Commission Joint Research Centre\tablefootnote{http://forobs.jrc.ec.europa.eu/products/gam/download.php}\\
\hline
Distance to Rivers & World Wildlife Fund \tablefootnote{http://hydrosheds.cr.usgs.gov/index.php}\\
\hline
Vegetation & NASA LTDR\tablefootnote{http://ltdr.nascom.nasa.gov/cgi-bin/ltdr/ltdrPage.cgi} \\
\hline
Carbon Storage & NASA JPL (\cite{saatchi_benchmark_2011})\tablefootnote{http://click.jpl.nasa.gov/Archive/carbon/ftpdata/carbon/datasets/} \\
\hline
Ecofloristic Zone Carbon Fractions & Oak Ridge National Laboratory (\cite{ruesch_new_2008}) \\
\hline
\end{tabular}
\end{table}

\section{Methods}
Significant progress has been made on methods which integrate spatial data (i.e., satellite information on forest cover) to quantify the causal impact of interventions (i.e., projects aimed at the prevention of deforestation) (\cite{nelson_effectiveness_2011}). These methods largely rely on propensity score and other matching-based methods to select ``control'' cases where no or limited intervention occurred, and match these with similar ``treatment'' cases at the sites of interventions (\cite{andam_measuring_2008}). We build on these approaches, implementing a geographically explicit two-stage Propensity Score Matching estimation strategy. This is motivated by the context of this analysis: specifically, we hypothesize that the impact of World Bank projects is geographically heterogeneous - i.e., a project in the Sahara is unlikely to have the same impact as one in the Amazon. In this section, we detail one approach which enables researchers to measure impacts in a way which flexibly incorporates geographic heterogeneity, and in the discussion highlight many limitations and possible extensions.

\subsection{Geographically Explicit Impacts}
First, we subset the data to cover World Bank projects from 2000 to 2010 due to limitations of our ancillary information, resulting in a total of 41306 World Bank project locations. From this, we remove data which does not have precise latitude and longitude information - leaving us with 19940 locations. Second, the area of influence within which we anticipate each World Bank project could plausibly have an impact on deforestation is calculated by examining the historic spatial distance at which forest cover is spatially correlated. To parameterize this distance, we calculate a Moran's I (\cite{getis_analysis_1992}) score at increasing distances, a metric that measures the degree of spatial autocorrelation for a given variable. We use this metric to estimate the distance at which spatial autocorrelation is no longer predominant for our outcome measure of forest cover, measured in 1999. We argue that this is a highly conservative estimate of the possible area of influence a project could have (i.e., we will tend to over-estimate the buffer size), as it represents the totality of historic spillovers up to the year 1999.
\par
For each of 12 distance bins (between 0 and 2,200 kilometers, in increments of approximately 180km), Moran's I is calculated following:

\begin{equation}
I_h = (\frac{N}{\sum_{i}^{N}\sum_{j}^{N}w_{ij}}) * ((\sum_{i}^{N}\sum_{j}^{N}w_{ij} * (X_{i}-\bar{x}) * \frac{X_{j} - \bar{x}}{\sum_{i}^{N}(X_{i}-\bar{x})})^{2})
\label{EQmoran}
\end{equation}

where \textit{h} represents each spatial bin, \textit{N} the number of spatial units, \textit{i} and \textit{j} are indexes for each unit, \textit{X} is the variable of interest, and \begin{math}W_{ij}\end{math} represents the weights matrix. In this application, the weights matrix is specified according to the bin (\textit{h}) being analyzed.  
\par
Once calculated, the distance at which Moran's I is equal to or less than .10 is identified, and used to parameterize a buffer around each project location. The projects locations are then subdivided into six different groups based on project sectors - Health, Environment, Education, Industrial, Infrastructure, and Other. For each sector group, the respective locations are further subdivided into three equally-sized monetary bins of ``low", ``medium", and ``high" based on dollars committed.
\par
Finally, for the high dollar value group, a geographically explicit propensity score model is fit. This is conducted following a three-stage process which is repeated for every high dollar value World Bank project location, resulting in a model for each of the project locations. In the first stage, a high dollar value World Bank project location is selected and all World Bank projects that fall within the distance threshold estimated according to the Moran's I are selected as the relevant ``subpopulation'' for that point. All points within the subpopulation are defined as treated or untreated pending their monetary value - units of observation with a high dollar value (in the upper 33\% for a given sector) are assigned as treated (1), while all projects in the  low bin (in the lower 33\% for a sector) are assigned an untreated value (0).
\par
In the second stage, all points within the subpopulation are matched according to a propensity score matching routine. Variables matched on can be seen in Table 1. The propensity scores are calculated once globally following a logit model:

\begin{equation}
logit \Bigg \{ {E} [P(T=1 | X_{1...k})] \Bigg \}= \beta_{0} + \sum_{k=1}^{k}(\beta_{k}*x_{k})
\label{EQpropensity}
\end{equation}

where \textit{T} is the treatment binary, and \begin{math}\beta_{k}\end{math} are the estimated coefficients for each covariate, \begin{math}x_{k}\end{math}.  
\par
The estimates from this equation are applied to each unit of observation within the subpopulation, and the differences between propensity scores across different units of observation are used to represent a univariate measure of similarity (extensive discussions of propensity score matching and it's application can be found in \cite{rubin_estimating_1997} and \cite{abbay_does_2015}). For the set of high dollar value locations within the area of influence, the optimal set of matched untreated units (without replacement) are identified using a nearest-neighbor optimization (\cite{ho_matchit:_2011}). This results in a dataset in which each treated unit is matched with the single control unit most similar to it, with units that have no meaningful comparison dropped from the analysis.
\par
In the third stage, a linear regression relationship is estimated between the outcome measure (the average LTDR NDVI value in the years after project implementation), the treatment binary, and all available covariates\footnote{A traditional, ordinary least squares model is fit using QR decomposition to promote the computational feasability of this approach, but the authors note other modeling approaches (SAR, GLM) may be more appropriate in some use cases.}:

\begin{equation}
y_i = \beta_{0} + \theta * T + \sum_{k=1}^{k}(\beta_{k}*x_{k}) + D_{p} + D_{s} + (\theta * D_{s})
\label{EQgwr}
\end{equation}

where \begin{math}y_{i}\end{math} represents the level of forest cover within each zone \textit{i}, \begin{math}\theta\end{math} represents the estimated impact of the treatment, \begin{math}D_{p}\end{math} represents a fixed effect for each paired observation, and \begin{math}D_{s}\end{math} represents a sector-specific fixed effect.
Every unit of observation \textit{n} has a zone \textit{i}, defined as all locations which fall within the distance calculated using Moran's I.
\par
This process is repeated for every unit of observation in the high dollar value subset.  
In some cases, insufficient matches or eligible cases existed to approximate the impact for a region; these units were omitted from the analysis.

\subsection{Estimating Carbon Sequestration}
Because the outcome measure examined (NDVI) is only a proxy for carbon, an additional step of modeling must be conducted to translate changes in NDVI into changes in estimated relative tonnes of carbon sequestered. To accomplish this, we employ a fixed-effects approach to account for the geographically variable relationship between NDVI and carbon (a heterogeneous relationship largely driven by different floral regimes across the globe). This relies on two datasets: an estimate of global vegetative carbon stocks representing the year circa 2000 (\cite{saatchi_benchmark_2011}), and ecofloristic zone information representing key geographic divisions of flora relevant for carbon (\cite{ruesch_new_2008}). Using this information in conjunction with LTDR NDVI from 2000, a fixed effect model is fit:

\begin{equation}
Carbon = \beta_{0} + \beta_{1} * NDVI + D_{ez}
\label{EQcarb}
\end{equation}

where \begin{math}D_{ez}\end{math} represents a fixed effect for each of 60 ecofloristic zones. The ecofloristic zone that each World Bank project exists in is then identified and used in conjunction with the impacts estimated in the geographically explicit methodology outlined above to estimate the relative carbon sequestration attributable to a given World Bank project location.

\section{Results}
First, we use the Moran's I measurements (eq. \ref{EQmoran}) to select a buffer radius to use in the estimation of each individual location. As Figure \ref{DDFig} illustrates, the distance-decay function of NDVI in 1999 follows an expected pattern, with spatial autocorrelation dropping off as distances increase. We use this information to select a buffer radius of 800 kilometers as our threshold (Moran's I ~= .10). For each unit of analysis we then draw a subpopulation of all locations which fall within the 800km radius.
\par
For each of these subpopulations, we match control and treatment cases on the basis of the propensity scores estimated in eq. \ref{EQpropensity}, following a nearest-neighbor matching strategy. A caliper of .25 is used to exclude poor matches, and after matching if a sufficient total of matches does not exist (less than 30 total matches), the unit is excluded from analysis and we move to the next subpopulation. 
\par
 After matching is conducted for each subpopulation, a regression is performed for that subpopulation following eq. \ref{EQgwr}. This results in 8399 locations which have adequate matches for estimation, or 47\% of all large scale projects' locations. For each of these models, we record all relevant information regarding standard errors and estimated coefficients\footnote{An online webmap was created to illustrate results to readers, but is omitted from this submission to facilitate blind peer review.}. The impacts estimated for each of these locations (\begin{math}\theta\end{math} and the sector-specific interaction term in eq. \ref{EQgwr}) are entered in to the fixed-effect model derived following eq. \ref{EQcarb}. This provides a regionally-specific estimate of the tonnes of carbon sequestered attributable to a World Bank project. A regional- and temporal- disaggregation of the results across all estimated projects can be seen in Figure \ref{result_fig}.
 
\begin{figure}[H]
\centering
 \includegraphics[width=0.45\textwidth]{pre_avg_NDVI_max_full.png}
\caption{Distance decay of NDVI values in 1999.}
\label{DDFig}
\end{figure}

\begin{figure}[H]
\centering
 \includegraphics[width=0.45\textwidth]{result_disag.png}
\caption{Results from estimates by region and time.}
\label{result_fig}
\vspace{10pt}
\end{figure}  

\section{Discussion and Conclusion}
The approach outlined in this document highlights a number of interesting findings, but is constrained by significant limitations on methodologic fronts. Of the key findings, we highlight the general improvement of World Bank projects over time, most notably in south Asia - a trend which could be reflective of functional environmental safeguards. However, we also highlight the significant geographic variation in this finding.  For example, development projects in India almost universally had relatively negative impacts on sequestration, while those in the Philippines had relatively positive impacts \begin{math}^{3}\end{math}. This is also evident on a region-by-region basis, as the negative trend line within the Middle East and North Africa highlights (see Figure \ref{result_fig}).  
\par
This approach has the benefit of contrasting World Bank locations to other locations at which it is known a World Bank project (albeit of a small magnitude) exists, and comparisons are conducted within projects that are - at least - known to be within the same sector. Further, by leveraging the geographic context in which projects exist, this approach has the potential to improve matches by providing pairs which are contextually similar - i.e., projects in dense forests are compared to other projects in dense forests; those near urban areas and contrasted to others near urban areas. Both of these attributes help to mitigate concerns over omitted variable biases, though come with drawbacks noted below. Lastly, by leveraging the geographically-explicit approach detailed here, each location receives an estimated impact. Thus, the geographic subpopulations generated in this approach provide unique insights into trends that may vary over space.
\par
Many opportunities exist to advance research which seeks to incorporate geographic data into models which causally identify impacts. First and foremost are the well known disadvantages to geographically weighted regression (GWR) approaches - namely spatial correlation in estimated coefficients, bias in standard error terms (\cite{wheeler_multicollinearity_2005}), and the necessity to define a weights matrix (i.e., in this piece we choose a Moran's I threshold of 0.1 to approximate a single threshold, but many alternative means for estimation of relevant thresholds exist).  
These factors limit the interpretation of the estimates calculated in this paper, specifically preventing insights into the significance of treatment impacts at any single project location.
Ongoing research is examining potential solutions to this problem - for example, leveraging the techniques of Seemingly Unrelated Regression (SUR) or the emergent causal machine learning approaches (see \cite{athey_recursive_2015}), but much of this research is currently nascent - and solutions appear to be extremely computationally intensive.
\par
A second limitation of this approach is in the matching strategy employed. We chose to contrast high-dollar value World Bank projects to low-dollar value World Bank projects, but outside of sectoral information have relatively little knowledge regarding the actual projects that were implemented at any given site. While we incorporate sectoral-specific fixed effects to ensure - to the degree possible - we are comparing  ``apples to apples", and further mitigate this problem by only selecting projects for which exact geographic information is available (thus omitting many broader, country-level initiatives that are rarely immediately associated with physical land change), the potential for bias due to poor comparison still exists. This is representative of a broader concern of any top-down approaches to impact evaluation, as there is frequently limited information on the characteristics of the project and relevant geographic contextual factors to include. Ongoing research into key characteristics of projects (i.e., beyond the number of dollars allocated and sectoral grouping, and including factors such as spatial correlation amongst covariates) seeks to mitigate these concerns, and provide increasingly better matches when top-down strategies are pursued.
\par
Despite these limitations, we believe this approach provides policymakers with a cost-effective approach to rapidly assess a very large portfolio of projects to identify "warning flags" or "bright spots". We do not suggest that such analyses take the place of traditional impact evaluation strategies, but rather argue that top-down analyses such as these can help better direct resources for more rigorous, in-situ assessments. Further, because we leverage satellite information which is regularly updated, such strategies could be applied not only to project evaluation, but also project monitoring.
\par
Following this, we argue that sub-national data can be helpful in the identification of geographically heterogeneous impact effects. This piece highlights this by examining the impact of large scale World Bank projects on carbon sequestration at a global scale, using a novel and publicly available data set of World Bank project locations\begin{math}^{1}\end{math}. We find that while these projects appear to have an overall positive effect, significant temporal and geographic variation exists which would be masked if single, aggregate estimates were examined. Finally, we argue for the importance of further research into methods to estimate geographically heterogeneous impacts effects.

%%%%%%%%%%%%%%%%%%%%%%%%%%%%%%%%%%%%%%%%%%
\vspace{6pt} 

%%%%%%%%%%%%%%%%%%%%%%%%%%%%%%%%%%%%%%%%%%
\acknowledgments{The authors would like to acknowledge the government of Sweden and the World Bank Independent Evaluation Group for partially funding this research.  This work was performed in part using computational facilities at the College of William and Mary which were provided with the assistance of the National Science Foundation, Virginia Port Authority, Virginia's Commonwealth Technology Research Fund, and the Office of Naval Research. The authors would also like to thank Scott Stewart, Alex Kappel, Zhonghui Lv, Doug Nicholson, Carrie Dolan, and Vinay Vijayan for their valuable contributions and insights.}

%%%%%%%%%%%%%%%%%%%%%%%%%%%%%%%%%%%%%%%%%%
\authorcontributions{D.R., J.T., G.B., M.L., and A.B conceived and designed the experiments; D.R., S.G., J.N., and R.M. analyzed the data, J.T. contributed data and materials, R.T., R.M. and D.R. wrote the paper and integrated feedback from numerous colleagues.}
%%%%%%%%%%%%%%%%%%%%%%%%%%%%%%%%%%%%%%%%%%
\conflictofinterests{The authors declare no conflict of interest.} 

%%%%%%%%%%%%%%%%%%%%%%%%%%%%%%%%%%%%%%%%%%
% Citations and References in Supplementary files are permitted provided that they also appear in the reference list here. 

\bibliographystyle{mdpi}
\bibliography{WBVFM_IntroPar}

%%%%%%%%%%%%%%%%%%%%%%%%%%%%%%%%%%%%%%%%%%
\end{document}

